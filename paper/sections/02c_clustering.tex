%!TEX root = ../paper.tex

\subsection{Clustering}
\label{sec:clustering}

After extracting all features and vectorizing them for each blog post in the database, we group them into clusters according to the similarity of their feature vectors.
To this end, we use a k-means algorithm which is a method from vector quantization.
We describe its functionality in Section~\ref{sec:impl_k-means}
Afterwards the resulting clusters are labelled, so that the user knows what kind of blog posts he can expect to find in a specific cluster.

% [TODO: discuss other algorithms?]

\subsubsection{Cluster Labelling}
\label{sec:cluster_labeling}
Labelling the resulting clusters helps the user to distinguish between them.
The labels indicate what kind of blog posts are in those clusters, so that the user gets an idea what to expect from blog posts in this cluster.

The current research in the field of cluster labelling focuses on topic-dependent labels.
Because we want to cluster by writing style rather than topic, we were unable to use those techniques.
Howver, trying to find universally accepted writing styles leads to labels like ``informative'' or ``opinion piece'', which merely represent the text's intention, not the author's personal style~\cite{lee2001genres}.
Thus, these types of labels were also not applicable for our use case.

Therefore, we implemented our own cluster labelling methods, which consider the most important features of a cluster.
These methods are described in section \ref{sec:impl_cluster_labeling}
