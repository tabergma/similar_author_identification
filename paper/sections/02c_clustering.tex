%!TEX root = ../paper.tex

\subsection{Clustering}
\label{sec:clustering}

After extracting all features and vectorizing them for each blog post in our data set, we group them into clusters according to the similarity of their feature vectors.
To this end, we use a k-means algorithm which is a method from vector quantization.
K-means is one of the most popular and simplest clustering algorithms.
Using a k-means algorithm the number of resulting clusters can be configured easily.

%TODO Genereller Satz fehlt: Wieso clustern wir? Was ist das Ergebnis des Clustering?

To run a k-means algorithm, the value $k$ has to be set beforehand.
$K$ represents the number of clusters and thus the number of similar writing styles, which should be determined.
Finding an appropriate value for $k$ depends highly on the use case.
One use case could be to find exact authors, so that one cluster, i.e. writing style group, would correspond to one author.
One cluster would only contain blog posts of one author.
The value $k$ should therefore be similar to the number of expected authors.
Another use case could be to find similar writing style groups.
One cluster would then contain blog posts from multiple authors having a similar writing style.
In this case $k$ should be chosen much lower than the actual number of authors.

\subsection{Cluster Labelling}
\label{sec:cluster_labeling}
Labelling the resulting clusters helps the user to distinguish between them.
The labels indicate what kind of blog posts are in those clusters, so that the user gets an idea what to expect from blog posts in this cluster.


A lot of current research in the field of cluster labelling focuses on topic-dependent labels.
Because we want to cluster by writing style rather than topic, we were unable to use those techniques.
However, trying to find universally accepted writing styles leads to labels like ``informative'' or ``opinion piece'', which merely represent the text's intention, not the author's personal style~\cite{lee2001genres}.
Thus, these types of labels were also not applicable for our use case.


%TODO Satz was wir jetzt gemacht haben.