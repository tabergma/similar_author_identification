%!TEX root = ../paper.tex

\subsection{Clustering}
\label{sec:clustering}

After extracting all features and vectorizing them for each blog post in our data set, we group them into clusters according to the similarity of their feature vectors.
Thus, the blog posts are grouped by their writing style into clusters.
One cluster represents one writing style, so that one cluster contains only blog posts with similar writing styles.


To this end, we use a k-means algorithm, which is a method from vector quantization.
K-means is one of the most popular and simplest clustering algorithms.
Using a k-means algorithm the number of resulting clusters can be configured easily.


To run a k-means algorithm, the value $k$ has to be set beforehand.
$K$ represents the number of clusters and thus the number of similar writing styles, which should be determined.
Depending on the value of $k$ the use case can differ.
Setting the value $k$ similar to the number of expected authors, the exact author should be found.
If $k$ is euqal to the number of authors, one cluster, i.e. writing style group, would correspond to one author.
So one cluster would contain only blog posts of one author.
Choosing a value much lower than the actual number of authors for $k$, would allow us to find similar authors.
In this case, one cluster would contain blog posts from multiple authors having a similar writing style.



\subsection{Cluster Labelling}
\label{sec:cluster_labeling}

Labelling the resulting clusters helps the user to distinguish between them.
The labels indicate what kind of blog posts are in those clusters, so that the user gets an idea what to expect from blog posts in this cluster.


A lot of current research in the field of cluster labelling focuses on topic-dependent labels.
Because we want to cluster by writing style rather than topic, we were unable to use those techniques.
However, trying to find universally accepted writing styles leads to labels, such as ``informative'' or ``opinion piece'', which merely represent the text's intention, not the author's personal style~\cite{lee2001genres}.
Thus, these types of labels were also not applicable for our use case.


Therefore, we implemented our own cluster labelling methods, which consider the most important features of a cluster.
These methods are described in Section~\ref{sec:impl_cluster_labeling}.