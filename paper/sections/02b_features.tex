\subsection{Features}
\label{sec:features}

An individual's writing style can be discerned by various attributes or features inherent to any written text.
Some of these features are topic-dependent, e.g. the count of occurrence for each existing word in the \textit{bag of words} model.
A text containing an above average number of occurrences of words, such as ``DNA'', ``sequence'', ``genome'', and ``mutation'' will most likely be from the field of genome research.
Thus, this feature will be similar for most texts from this domain and therefore is topic-dependent.


However, since we want to classify blog posts independent of topic, we can utilize only topic-independent features.
We represent these features as real valued numbers, normalized to be between 0  and 1.
Each attribute of a text can be represented by multiple features, for example the frequency of punctuation characters can have one feature per punctuation character, denoting that character's frequency as a fraction of all characters.
After calculating all features, the combination of them is interpreted as a vector for each text and then passed on to the clustering or classification step.


We have selected a number of these topic-independent features from~\cite{madigan2005author} and~\cite{narayanan2012feasibility} and added a few of our own, which are more tailored towards blog posts.


Our emoticon feature is one that we selected to specifically target the blogging realm.
Since emoticons hardly ever appear in traditional written works, they have not been used when trying to identify authors outside of the internet.
In the blogosphere, however, there are some users who utilize lots of emoticons and others that don't use them at all.
The types of emoticons used also differ from person to person, which is why we decided to implement that feature as a binary value of a certain emoticon appearing in the text or not.
A listing of the emoticons our system uses can also be found in Appendix~\ref{sec:app_emoticons}.


We also added an \textit{upper case} feature, which calculates the fraction of all words written in all capital letters since we found that some authors use capitalization as a form of emphasis or to delimit direct speech and the speaker.
Since only some writers use capitalization in this way and those that do use it to varying degrees, we found it to be a useful feature.


While we re-used many features, which have been previously described in other work, we left out some of them as they were either topic-dependent or not applicable for the blog domain.
One example for the second case would be a greeting or signature feature~\cite{de2001mining}.
While some writers do use individual greeting or salutation phrases, far from all of them do.
This means that, while we might receive good results for those that use them, such features would be counterproductive for all other authors as their first or last lines of a blog entry would vary tremendously from post to post.
%Additionally, they are rather useless for determining the actual style of someone's writing.
Using a list of known greetings or signatures to determine if one was used might be an option, but we decided against doing so, because of the sheer amount of individual phrases we came across during our research on the topic.

A list of all the feature types we initially used can be found in Section~\ref{sec:impl_features}.
