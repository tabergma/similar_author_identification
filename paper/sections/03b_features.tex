\subsection{Features}
\label{sec:features}

An individual's writing style can be discerned by various attributes or features inherent to any written text.
Some of these features are topic-dependent, e.g. the count of occurrence for each existing word in the ``Bag of Words'' model.
A text containing an above average number of occurrences of words like ``DNA'', ``sequence'', ``genome'', and ``mutation'' will most likely be from the field of genome research.
Thus, this feature will be similar for most texts from this domain and therefore is topic-dependent.


However, since we want to classify blog posts independent of topic, we can only utilize topic-independent features.
We represent these features as real valued numbers, normalized to be between 0 and 1.
Each attribute of a text can be represented by multiple features, for example the frequency of punctuation characters can have one feature per punctuation character, denoting that character's frequency as a fraction of all characters.
After calculating all features, the combination of them is interpreted as a vector for each text and then passed on to the machine learning or classification step.


We have selected a number of these topic-independent features from~\cite{madigan2005author} and added a few of our own, which are more tailored towards blog posts, e.g. emoticons.
A list of the feature types we initially used can be found in Table~\ref{tab:featureTable} along with their respective individual feature count.


One of the features we utilized, the FunctionWord feature might appear to clash with our desire to create a topic-independent algorithm which does not use a Bag of Words model.
However, function words are words with little or no lexical meaning, mainly used to create the grammatical structure of a sentence.
Thus, they are actually topic-independent and the frequency of their usage has been successfully utilized to identify authors [TODO: Reference!].
We also tried to achieve topic-independence for our list of common abbreviations which we used in the same way.


Our emoticon feature is one that we selected to specifically target the blogging realm.
Since emoticons hardly ever appear in traditional written works, they have not been utilized when trying to identify authors outside of the internet.
In the blogosphere, however, there are some users who utilize lots of emoticons and others that don't use them at all.
The types of emoticons used also differ from person to person which is why we decided to implement that feature as a binary value of a certain emoticon appearing in the text or not.
%[TODO: If we need to fill room: Explain some features in detail (vocabulary richness, PoS tags)]


After evaluating our features (see Section~\ref{sec:evaluation_clustering}) we decided to drop the Prefix/Suffix feature and keep all others.

\begin{table}[h]
    \begin{center}
    \begin{tabular}{p{2.6cm}|p{8.2cm}|p{1.2cm}}
    Feature					& Description																	& Count				\\ \hline
    BlankLine				& 1 divided by number of Blank Lines in the text								& 1					\\ \hline
    CapitalLetter			& Capital letters divided by all letters										& 1					\\ \hline
    Emoticon				& Boolean value for occurrence of known emoticons								& 9					\\ \hline
    FunctionWord			& Fraction of words that are a known function word. [TODO: List in appendix?]	& 280[DE] 280[EN]	\\ \hline
    Abbreviations			& Words that are a known abbreviation [TODO: List in appendix?]					& 53[DE] 53[EN]		\\ \hline
    NumberCharacter			& Fraction of characters that are numeric characters (0-9)						& 1					\\ \hline
    Paragraph				& Number of Paragraphs divided by average Paragraph								& 1					\\ \hline
    PoSTag					& Fraction of words that have a certain Part-of-Speech Tag~\cite{toutanova2000enriching}                                                                                                        & 55[DE] 46[EN]  	\\ \hline
    PostLength				& 1 divided by the number of characters											& 1					\\ \hline
    PrefixSuffix			& 2-letter Prefixes/Suffixes divided by total number of Prefixes/Suffixes		& 676 			 	\\ \hline
    Punctuation Character	& Fraction of characters that are a known punctuation character [TODO: List in appendix?]
                                                                                                            & 11				\\ \hline
    SentenceLength			& 1 divided by the average sentence length										& 1 				\\ \hline
    singleOccurring Word	& Words that only occur once divided by all distinct words						& 1					\\ \hline
    UpperCaseWord			& Words that are fully in upper case divided by all words						& 1					\\ \hline
    WordFrequency			& 1 divided by the average number of occurrences per word						& 1					\\ \hline
    WordLength				& 1 divided by the average word length											& 1					\\
    \end{tabular}
    \end{center}
	\caption{Feature Table. Implemented features with descriptions and the count of individual features they contribute to the feature vector.}
	\label{tab:featureTable}
\end{table}
