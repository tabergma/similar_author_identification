%!TEX root = ../paper.tex
\section{Conclusion and Future Work}
\label{sec:conclusion}


Nowadays, everyone can write a blog post fairly easily.
Thus, the amount of blog posts and authors increases quickly, making the task of identifying an exact author hard to conclude.
In this paper, we present an approach of dividing a set of blog posts into different groups, each group representing an individual writing style and therefore similar authors.
Thus, we are reducing the potential candidates by an order of magnitude or more and we can aid the user in getting much closer to the correct author.


We tested and evaluated multiple features for identifying different writing styles.
Having a feature, which results in hundreds of subfeatures while having only a few other features, adulterates the results immensely.
We use a k-means algorithm to cluster the blog posts based on the selected features.
Our evaluation on a small test data set shows that our k-means algorithm has a f-measure of $61.64\%$.
To determine a cluster, i.e. writing style, of a new blog posts, we examined different classification methods.
K-nearest neighbor performed slighty worse ($97.77\%$) than a support vector machine ($98.89\%$).


Our approach of finding blog posts similar to one selected blog post, e.g. the blog posts, which are in the same cluster, can be used to find blog posts similar to one a user likes.
Since we can search across topics, a new type of recommendation engine could be created to recommend blogs based on writing style alone.
Our system could be employed on blogging websites extending traditional topic-based recommender systems to also consider writing style.
This might improve recommendation quality or even find good matches for blogs previously considered to be not interesting to a user, thus increasing traffic and advertising revenue.


Our program's results and its usability could be improved in various ways: Adding more languages to our program’s repertoire would greatly increase its scope.
This would merely require adjusting the calculation of some of the features by adding an appropriate Part-of-Speech tagger and a function word list as well as new tests on the performance and feasibility of each feature.


Utilizing a larger and more diverse data set with a proper gold standard could allow for improved evaluation of our program’s ability to find exact matches for unknown authors.
This would also allow for a more in-depth discussion of which additional features to use and how to weight them.


Additional work could also be put into our clustering step by adding an additional phase prior to the k-means algorithm to attempt to find the optimal number of clusters for it to use.
Alternatively, evaluating the feasibility of other clustering algorithms is also an option.


Furthermore, our approach could be expanded by additional features, which helps to identify for example the gender, the occupation, or the age of the authors.
Maybe woman use more adjectives than men and therefore blog posts can be divided into gender groups.
A group would no longer just represent the writing style of blog posts but would also give the user information about what kind of author wrote those blog posts.


Another idea is to use our approach to find exact authors.
One possibility is to first identify the writing style group of a blog posts and then search for the exact author in that group.
In that way the number of possible matching authors is reduced significantly.
Another possibility is to set the number of groups similar to the number of authors.
In the best case one author would be assigned to one group containing only blog posts of this author.
The labels for the groups are then no longer the writing style characteristics but the author names.
