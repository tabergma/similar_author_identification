\subsection{Classification}
\label{sec:classification}

After we came up with clusters and cluster labels for all blog posts in a database, we need a method to classify new blog posts. % e.g. so new clustering is not needed for every single blog post
To achieve this we calculate the same features vectors for the single new blog post, as were extracted for the blog posts in the original clustering.
Based on these feature vectors we can now decide to which cluster the blog post belongs.


The first naive idea that came into our mind was to calculate the euclidean distance from the blog post to the center points of the clusters.
The resulting cluster would then be the one with the lowest distance.
While this method is computationally simple and fast, the task of classification is a well-researched machine learning problem.
Therefore we looked at more sophisticated approaches that were used for similar problems before.
We applied two of those methods, k-nearest Neighbor [TODO: reference!] and a Support Vector Machine [TODO: reference!].

