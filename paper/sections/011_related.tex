%!TEX root = ../paper.tex

\section{Related Work}
\label{sec:related}
% \begin{itemize}
% 	\item Who inspired your work?
% 	\item Which aspect do you improve?
% 	\item What are the basic foundations?
% \end{itemize}


% topic independence, small number of documents
There is related work in the fields of author identification, based only on a small number of documents \cite{de2001mining}.
This paper deals with email content, written by three authors and spread over three topics: movies, food, and travel.
To allow author discrimination while covering multiple topics, the used features have to be topic-independent.
This paper furthermore states, based on \cite{corney2001identifying}, that about 20 documents per author should be sufficient for a satifying categorisation performance.


A different paper, that focuses on a larger scale with 27,000 documents can be found under \cite{madigan2005author}. A significant difference compared to \cite{de2001mining} was, that the number of features used was much higher and more diverse.


However the number of documents dealt with in \cite{narayanan2012feasibility} was much closer to our desired data set size.
The authors used 2.4 million blog posts from about 100,000 blogs with roughly the same number of authors.
The machine learning approaches used, were k-Nearest Neighbor, Naive Bayes, Support Vector Machine, and Regularized Least Squares Classification.
